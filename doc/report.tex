\documentclass[conference]{acmsiggraph}

\usepackage{mdwlist}
\usepackage{graphics}
\usepackage{graphicx}
\usepackage{listings}
\usepackage{enumitem}
\usepackage{amsmath}
\usepackage[utf8]{inputenc}

\newcommand{\onecol}[3]{
  \begin{figure}
    \centering
    \includegraphics[width=\columnwidth]{#1}
    \vspace{0.0in}%??
    \caption{#3}
    \label{#2}
  \end{figure}
}

\newcommand{\twocol}[3]{
  \begin{figure*}
    \centering
    \includegraphics[width=5.5in]{#1}
    \vspace{0.1in}%??
    \caption{#3}
    \label{#2}
  \end{figure*}
}

%%%%%%%%%%%%%%%
% Information %
%%%%%%%%%%%%%%%

\title{Animating Smoke Using Particle Systems And Fluid Dynamics}

\author{
  Andre Philipp\thanks{e-mail:philipp.andy@gmail.com}
  \qquad
  Andy Spencer\thanks{e-mail:andy753421@ucla.edu}
  \qquad
  Haakon Garseg Mørk\thanks{e-mail:haakongmork@gmail.com}
  \\
  University of California, Los Angeles \\
  CS274C Computer Animation
}

\pdfauthor{Andre Philipp, Andy Spencer, Haakon Garseg Mørk}

\keywords{partical systems, fluid dyamics, smoke, fire, webgl}

%%%%%%%%%%%%
% Document %
%%%%%%%%%%%%

\begin{document}

%%%%%%%%%%%%%%
% Title Page %
%%%%%%%%%%%%%%

\maketitle

\begin{abstract}

In this project we explored the possibilities of using particle systems and
fluid dynamics to animate smoke.

Particle systems are often used in computer animation to animate different types
of natural phenomenons like fire, fluids, fog, smoke, and many times a
combination that can for instance be an explosion. Particle systems is also use
to animate a huge amount of other phenomenons. In these examples given above,
the particle system typically have an emitter, which is a source for the
particles. The particles itself holds some parameters that is used to simulate
the behaviour of each particle (e.g. velocity vector, color/image, lifetime,
etc.). 

In computer animation fluid dynamics is used to deal with fluid flow. This is
necessary to simulate a realistic behaviour of fluids. Fluid dynamics is a
subdiscipline of fluid mechanics, and has several sub disciplines itself like
hydrodynamics and aerodynamics. 

\end{abstract}

\keywordlist

%%%%%%%%%%%%
% Content %
%%%%%%%%%%%%

\section{Software Architecture}

WebGL is a JavaScript API for rendering 3D graphics in a web browser, released
in 20011. It is supported by all of the most popular web browsers. We choose to
use WebGL, because it is very accessible and requires no plugins, or software
other that the web browser.

Three.js is a cross-browser JavaScript API that provides a simple interface for
3D graphics in web browsers. It uses WebGL as an underlying API, and is one of
the most popular libraries for WebGL programming. Particle System support is
included in the original version of the library. We chose to use Three.js
because we wanted to focus on the animation part of the project rather than the
graphics part, and Three.js provides a level of indirection that suited our
needs.

\section{Fluid Dynamics Simulation}

Our fluid dynamics simulation is based on the work "Real-Time Fluid Dynamics for
Games" by Jos Stam. We extended the algorithms to three dimensions and modified
the data structures to be more suitable for implementation using JavaScript. In
the density function was implemented using an explicit method due to our use of
particle systems for smoke simulation.

\subsection{Velocity function}

Our velocity functions consists of diffuse, project, advect, and project steps.
Projection is performed twice, once before and once after the advect step. 

During diffuse the velocity field is modified by viscus forces. However, air can
generally be considered inviscus (todo: cite), therefore the diffuse step is not
needed during a smoke simulation.

During advection, the location of each velocity cell is back propagated in time
to determine the new velocities at each point. This step causes the velocity
field to lose the mass conceiving property which could lead to a convergence or
divergence of the smoke particles and an unrealistic simulation.

As air at relatively slow velocities can be treated as incompressible (todo:
cite), the mass conceiving property of the velocity must be maintained. This is
done by the project step which modifies to velocity field to remove divergences
and convergences.

\subsection{Density function}

Normally fluid simulations involve computing a density function on rectangular
grid. The density at each point represents the number of particles contained
within that space. This computation is similar to velocity function computations
but does not require a projection step. Furthermore, the diffusion step cannot
be left out of the density function.

However, for our simulation, instead of computing a density function we computed
the position of a large number of particles directly. This eliminates a large
amount of complexity in the algorithm and also simplifies rendering because the
particle positions are already known.

In place of the advection step move each smoke particle individually. Because
the resulting particles are not arranged on a rectangular gird we can perform a
simpler forward propagation rather than the reverse propagation used by
advection. Additionally we add random motion to each particle to simulation
diffusion of the smoke particles.

\subsection{Linear interpolation}

Linear three dimensional interpolations were used whenever velocity are
calculated for points in the velocity field. These include: the velocity of each
particle while moving the particles forward in time, starting location of the
velocities during velocity advection, and the position of the cursor when
interacting with the user.

\subsection{Boundary conditions}

Our simulation does not implement boundary conditions on the velocity field, and
smoke particles that move outside the simulated area are removed from the
simulation.

\section{Rendering}

In the real world, smoke is a collection of airborne solid and liquid
particulates and gases. It would not be feasible to simulate and render a
realistic number of particles for a WebGL real time animation like we did.
Fortunately, a lot can be done in the rendering stage to improve the final
result that ends up on the screen.

To deal with the high number of particles in the composition of real smoke, we
added a texture to the particles that makes each particle look like a set of
particles rather than a single particle, that way we are able to generate fewer
particles and still get a rich look. 

We found that when the forces acting on particles are not too complex, the
assumption that groups of particles will stick together does not result in
obvious unrealistic movement of the smoke. The particles opacity is adjusted to
be below 100\%, and a blending effect is applied to avoid a fragmented look, and
make the particles blend smoothly together.

\section{User Interaction}

Several forms of user interaction were included in our simulation. The user
interface provides the ability to start and extinguish fires on the ground,
create and stop wind in the air, and clear the smoke created by the fires.

\subsection{Fires}

Fires can be started by the user anywhere on the ground of the simulation. Each
fire consists of the fire location and a heat value. The heat from the fire is
added as an external force into the fluid dynamics simulation. Additionally, at
each timestep the fire will produce some number of smoke particles. The smoke
particles cleared from the simulation at the users request. 

todo: describe smoke particle hidingh

\subsection{Wind}

Wind can be added by the user though mouse interactions. Wind is a temporary
change in the velocity field and is propagated by the fluid dynamics equations
and interact with the heat and some particles produced by the fires. The
velocity field can also be reset to zero at any time.

\section{Conclusion}

Gas, is one of the most interesting states of matter in the context of Computer
Animation. The complexity and diversity of smoke, makes it hard to do accurate
physics based simulation. 

\section{Future Work}

The are many ways to improve the smoke simulation, the rendering could be
improved with shaders, better textures and more details in the smoke material.
Considering improvements on the animation parts only, we would like to improve
the following:

\subsection{Collision Detection}

We would like to add collision detection to the particle system to animate smoke
interacting with other objects in the environment. We do not think collision
detection between particles is a natural next step for a smoke animation as we
see overlapping particles a feature to simulate different densities of smoke.

\subsection{Fire/flames}

The focus of the project was on smoke and smoke animation, but smoke often
occurs in the context of a fire, and it would be an interesting next step to
include a fire animation as the source of the smoke.

\subsection{Better Equations?}

\section{References}

\begin{tabular}{l l}
  Smoke    & http://en.wikipedia.org/wiki/Smoke   \\
  Three.js & https://github.com/mrdoob/three.js/  \\
  WebGL    & https://www.khronos.org/webgl/       \\
\end{tabular}

\url{http://www.dgp.toronto.edu/people/stam/reality/Research/pdf/GDC03.pdf}

\url{http://www.autodeskresearch.com/pdf/ns.pdf}

\url{http://nerget.com/fluidSim/}

%%%%%%%%%%%%%%
% References %
%%%%%%%%%%%%%%

\bibliographystyle{acmsiggraph}
\bibliography{paper}

\end{document}
